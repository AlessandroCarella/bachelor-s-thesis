\chapter*{Abstract}

Il testo di tesi mira a identificare gli stati d’animo delle persone attraverso l’analisi di immagini o video che le riprendono.

Come attestato da diversi studi, gli stati d’animo incidono notevolmente sulle performance individuali, ed è per questo utile avere delle informazioni a riguardo col fine di rimodulare le attività e migliorare l’esperienza del singolo.

Lo studio trova applicazione sia in contesti di e-learning che di lavoro, o in qualsiasi altro contesto dove sia possibile ottenere delle riprese.

Un esempio pratico di un applicativo che usufruisce dei risultati del mio studio è l’utilizzo di questo, da parte di un docente, durante una lezione; l’insegnante, avendo cognizione dello stato d’animo dei suoi studenti, ha la possibilità di adattare lo stile di insegnamento e rendere più agevole la fruizione della lezione per i suoi alunni.